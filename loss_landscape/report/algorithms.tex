%!TEX root=main.tex
\section{Main Algorithms and Results}\label{sec:algorithms}




We are now ready to introduce our main algorithms and results for the unknown transition case, with either full-information or bandit feedback.
The complete pseudocode is shown in \pref{alg:bobw_framework},
which is built with two main components: a new framework to deal with unknown transitions and adversarial losses (important for incorporating our loss-shifting technique), and special regularizers for \ftrl.
We explain these two components in detail below.

\paragraph{A new framework for unknown transitions and adversarial losses}
When the transition is unknown, a common practice (which we also follow) is to maintain an empirical transition along with a shrinking confidence set of the true transition, usually updated in some doubling epoch schedule.
More specifically, a new epoch is started whenever the total number of visits to some state-action pair is doubled (compared to the beginning of this epoch), thus resulting in at most $\order\rbr{|S||A|\log T}$ epochs.
We denote by $i(t)$ the epoch index to which episode $t$ belongs. 
At the beginning of each epoch $i$, we calculate the empirical transition $\bar{P}_i$ (fixed through this epoch) as: 
\begin{equation}
\label{eq:empirical_mean_transition_def} 
\bar{P}_i(s'|s,a) = \frac{ m_i(s,a,s')  }{m_i(s,a)}, \quad \forall (s,a,s') \in S_k \times A \times S_{k+1}, \; k = 0,\ldots L-1, 
\end{equation}
where $m_i(s,a)$ and $m_i(s,a,s')$ are the total number of received visits to $(s,a)$ and $(s,a,s')$ accordingly prior to epoch $i$. 
{\color{magenta}
In other words, $m_i(s,a)$ is formally defined as 
\[
m_i(s,a) = \sum_{t=1}^{t_i - 1} \Indt{s,a} \cdot \ind_t^s  
\]
where $t_i$ is the epoch starting time, $\ind_t^s = \Ind{ t + d^t < s  }$ is the indicator of whether the tuple $(t, U_t, \{\ell_t(s_t^k, a_t^k)\}_{k<L})$ is revealed to learner prior to episode $s$.  We define  $m_i(s,a,s')$ similarly. 

On the other hand, we denote by $n_i(s,a)$ and $n_i(s,a,s')$ the underlying total number of visits to $(s,a)$ and $(s,a,s')$ prior to epoch $i$, that is, 
\[
n_i(s,a) = \sum_{t=1}^{t_i - 1} \Indt{s,a}, \quad n_i(s,a,s') = \sum_{t=1}^{t_i - 1} \Indt{s,a,s'}. 
\]
} 

\footnote{When $m_i(s,a) = 0$, we simply let $\bar{P}_i(\cdot|s,a)$ be an arbitrary distribution.}
The confidence set of the true transition for this epoch is then defined as
\begin{equation*}
%\label{eq:confidence_set_def}
\calP_i = \cbr{ \widehat{P}: \abr{ \widehat{P}(s'|s,a) - \bar{P}_i(s'|s,a) } \leq B_i(s,a,s'), \; \forall (s,a,s')\in S_k \times A \times S_{k+1}, k < L },
\end{equation*} 
where $B_i$ is Bernstein-style confidence width (taken from~\cite{jin2019learning}):
\begin{equation}
\label{eq:confidence_width_def} 
B_i(s,a,s') = \min\cbr{  2 \sqrt{\frac{\bar{P}_i(s'|s,a)\ln \rbr{\frac{T|S||A|}{\delta}}}{m_{i}(s,a)}} + \frac{14\ln\rbr{\frac{T|S||A|}{\delta}}}{3m_{i}(s,a)}, \; 1}
\end{equation}
for some confidence parameter $\delta \in (0,1)$.
As~\citep[Lemma~2]{jin2019learning} shows, the true transition $P$ is contained in the confidence set $\calP_i$ for all epoch $i$ with probably at least $1-4\delta$.

When dealing with adversarial losses, prior works~\citep{rosenberg19a, rosenberg2019online, jin2019learning, lee2020bias} perform \ftrl (or a similar algorithm called Online Mirror Descent) over the set of all plausible occupancy measures $\Omega(\calP_i) = \{q \in \Omega(\widehat{P}): \widehat{P} \in \calP_i\}$ during epoch $i$, which can be seen as a form of optimism and encourages exploration.
%This framework, however, does not allow us to apply the loss-shifting trick discussed in \pref{sec:loss_shifting} --- indeed, our key shifting function \pref{eq:loss_shift_function} is defined in terms of some fixed transition $\bar{P}$, and the required invariant condition on $\inner{q, g_\tau}$ only holds for $q \in \Omega(\bar{P})$ but not $q \in \Omega(\calP_i)$.

Inspired by this observation, we propose the following new approach.
First, to directly fix the issue mentioned above, for each epoch $i$, we run a new instance of \ftrl simply over $\Omega(\bar{P}_i)$.
%This is implemented by keeping track of the epoch starting time $t_i$ and only using the cumulative loss $\sum_{\tau=t_i}^{t-1}\hatl_\tau$ in the \ftrl update (\pref{eq:epoch_FTRL}).
Therefore, in each epoch, we are pretending to deal with a known transition problem. %making the same loss-shifting technique discussed in \pref{sec:loss_shifting} applicable.

However, this removes the critical optimism in the algorithm and does not admit enough exploration.
To fix this, our second modification is to feed \ftrl with optimistic losses constructed by adding some (negative) bonus term, an idea often used in the stochastic setting.
More specifically, we subtract $L \cdot B_i(s,a)$ from the loss for each $(s,a)$ pair, where $B_i(s,a) = \min\big\{1, \sum_{s' \in S_{k(s)+1}} B_i(s,a,s')\big\}$; see \pref{eq:adjusted_loss}.
In the bandit setting, note that the importance-weighted estimator is no longer applicable since the transition is unknown (making $q_t$ also unknown), and~\citep{jin2019learning} proposes to use $\frac{\ell_t(s,a) \cdot \Indt{s,a}}{u_t(s,a)}$ instead, where $\Indt{s,a}$ is again the indicator of whether $(s,a)$ is visited during episode $t$, and $u_t(s,a)$ is the so-called upper occupancy measure defined as
\begin{equation}\label{eq:UOB}
u_t(s,a) = \max_{\widehat{P} \in \calP_{i(t)}} q^{\widehat{P}, \pi_t}(s,a)
\end{equation}
and can be efficiently computed via the \textsc{Comp-UOB} procedure of~\citep{jin2019learning}.
Our final adjusted loss estimator is then $\hatl_t(s,a) = \frac{\ell_t(s,a) \cdot \Indt{s,a} }{u_t(s,a)}   - L \cdot B_{i}(s,a)$.
In our analysis, we show that these adjusted loss estimators indeed make sure that we only underestimate the loss of each policy, which encourages exploration.

{\color{magenta}
With delayed feedback, the loss function $\hatl_t$ is not revealed to the learner immediately but at the end of episode $t+d^t$. To this end, we denote the loss estimator  as 
\begin{align*}
\hatl_t^h(s,a) =   \frac{\ell_t(s,a) \cdot \Indt{s,a} \cdot \ind_t^h }{u_t(s,a)}   - L \cdot B_{i}(s,a)
\end{align*}
as the loss estimator of $\ell_t$ at episode $h$. 
}

With this new framework, it is not difficult to show $\sqrt{T}$-regret in the adversarial world using many standard choices of the regularizer $\phi_t$ (which recovers the results of~\citep{rosenberg19a, jin2019learning} with a different approach).
To further ensure polylogarithmic regret in the stochastic world, however, we need some carefully designed regularizers discussed next.

\DontPrintSemicolon 
\setcounter{AlgoLine}{0}
\begin{savenotes}
\begin{algorithm}[t]
	\caption{Best-of-both-worlds for Episodic MDPs with Unknown Transition}
	\label{alg:bobw_framework}
	
	\textbf{Input:} confidence parameter $\delta$.
	
	\textbf{Initialize:} epoch index $i=1$ and epoch starting time $t_i = 1$.
	
	\textbf{Initialize:} $\forall (s,a,s')$, set counters $m_1(s,a) = m_1(s,a,s') = m_0(s,a) = m_0(s,a,s') = 0$.
	
	\textbf{Initialize:} empirical transition $\bar{P}_1$ and confidence width $B_1$ based on \pref{eq:empirical_mean_transition_def} and \pref{eq:confidence_width_def}. 
		
	\For{$t=1,\ldots,T$}{
		{\color{magenta}
		Let $\phi_t$ be \pref{eq:Tsallis} for bandit feedback, and compute
	   \begin{equation}\label{eq:epoch_FTRL}
	    \widehat{q}_t = \argmin_{q\in \Omega\rbr{\bar{P}_i}}  \inner{q,  \sum_{\tau=t_i}^{t-1}\hatl_\tau^t} + \phi_t(q), 
	    \end{equation}
	    with adjusted loss estimator $\hatl_\tau^t$ definfed as
	    \begin{equation}\label{eq:adjusted_loss}
	    \hatl_\tau^t(s,a) = \frac{\ell_\tau(s,a) \cdot \ind_\tau(s,a)\cdot \ind_\tau^t  }{u_\tau(s,a)}  - L \cdot B_{i}(s,a), 
	    \end{equation}
	    where $B_i(s,a) = \min\big\{1, \sum_{s' \in S_{k(s)+1}} B_i(s,a,s')\big\}$, $\Indt{s,a} = \Ind{\exists k, (s,a)=(s_k^{t}, a_k^{t})}$, 
	    and $u_t$ is the upper occupancy measure defined in \pref{eq:UOB}.}
         
         
		Compute policy $\pi_t$ from $\widehat{q}_t$ such that
		$\pi_t(a|s) \propto \widehat{q}_t(s,a)$.\footnote{If $\sum_{b\in A} \widehat{q}_t(s,b)=0$, we let $\pi_t$ to be the uniform distribution.}
		
		Execute policy $\pi_t$ and obtain trajectory $(s_k^{t}, a_k^{t})$ for $k = 0, \ldots, L-1$.

		
		Increment counters: for each received tuple $\rbr{h, U_h, \{\ell_h(s_k^h,a_k^h)\}_{k<L} }$ that $h+d^h = t$, 
		$m_i(s_k^h,a_k^h,s_{k+1}^h) \overset{+}{\leftarrow} 1, \; m_i(s_k^h,a_k^h) \overset{+}{\leftarrow} 1$.\footnote{We use $x \overset{+}{\leftarrow} y$ as a shorthand for the increment operation $x \leftarrow x + y$.}  \\
		
		\If(\myComment{entering a new epoch}){$\exists (s,a), \  m_i(s, a) \geq \max\{1, 2m_{i-1}(s,a)\}$}
		{
                Increment epoch index $i \overset{+}{\leftarrow} 1$ and set new epoch starting time $t_i = t+1$.  
		
		 Initialize new counters: $\forall (s,a,s')$,  
		$m_i(s,a,s') = m_{i-1}(s,a,s') , m_i(s,a) = m_{i-1}(s,a)$.
		
                Update empirical transition $\bar{P}_i$ and confidence width $B_i$ based on \pref{eq:empirical_mean_transition_def} and \pref{eq:confidence_width_def}. 
}
	}
\end{algorithm}
\end{savenotes}

\paragraph{Special regularizers for \ftrl}
Due to the new structure of our algorithm which uses a fixed transition $\bar{P}_i$ during epoch $i$, the design of the regularizers is basically the same as in the known transition case.
Specifically, in the bandit case, we use the same Tsallis entropy regularizer: {\color{magenta}
\begin{equation}\label{eq:Tsallis}
\phi_t(q) =  -\sqrt{t - t_{i(t)}+1}\underbrace{\sum_{s \neq s_L} \sum_{a\in A} \sqrt{ q(s,a) } }_{ \text{Tsallis Entropy}  } + \frac{1}{\eta_t} \underbrace{\sum_{s\neq s_L} \sum_{a\in A} q(s,a)\ln q(s,a) }_{ \text{Shannon Entropy}  } + \beta  \underbrace{\sum_{s\neq s_L} \sum_{a\in A} \ln \frac{1}{q(s,a)}}_{ \text{Log Barrier}  },
\end{equation}
where $\eta_t = TODO $ and $\beta = 128L^4$.}
Similarly to~\citep{jin2020simultaneously}, the small amount of log-barrier in the second part of \pref{eq:Tsallis} is used to stabilize the algorithm.


\subsection{Main Results}\label{sec:results}

\begin{theorem}\label{thm:main_delayed}
\end{theorem} 



